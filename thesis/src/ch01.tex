\chapter{绪论}
\section{研究背景}
本论文受到国家自然科学基金面上项目《基于DBN协同建模的中文及跨语种语音结构事件检测研究》(编号 61175018)的资助,旨在研究利用非参数贝叶斯模型对广播新闻故事分割任务和音素分割任务进行建模和分析。
\subsection{非参数贝叶斯模型}
目前,概率模型已经成为了机器学习领域中的重要方法。许多不同领域的问题在利用概率模型进行抽象描述时,只是在特征表示上有一些区别,而本质上是一致的。因此,概率模型的一般性质和算法的研究非常重要。20世纪以来,概率论、数理统计和最优化理论等领域的发展为此提供了充分的基础,目前许多的概率模型理论都是将这些经典理论在机器学习这个主题下进行重新的解释。近年来,Koller和Jordan等人更是从更加宏观的角度对概率建模方法进行了研究,其发展出的图模型理论,为不同领域的各类问题提供了一套统一的概率建模方法和推断框架\cite{koller:2009,jordan:2003introduction,wainwright2008graphical}。

利用概率模型进行推断的结果是一个概率分布,这给模型带来了一定的鲁棒性,也更符合人们解释现象时的行为。然而,由于这个推断的结果是和模型的参数相关的,必须考虑模型的参数学习问题。对于模型的参数,一般有两种处理方法,一是利用最大似然法得到参数的值,然后固定该参数值进行推断。另一种则是为参数增加先验分布,得到参数的一组分布,当进行推断时,对参数所有取值下的推断结果进行加权,这一方法也叫做贝叶斯方法。无论采用哪一种方法,都需要固定参数的个数进行参数学习。然而,对于许多问题,其模型对于参数的个数是敏感的,如果参数的个数设置不当,模型的复杂度和数据不一致,则会使得模型难以有效的建模数据。比如对于聚类问题,当模型的聚类个数和数据的真实聚类个数不等时,得到的结果就无法令人满意。对于如何选择合适的参数个数,通常称为模型选择\cite{kohavi1995study,burnham2002model,hastie2009elements}问题。对于规模较小的问题,可以用人工实验来进行,但是也要耗费大量的人力物力,而对于许多实际任务,比如对于互联网海量数据的分析任务,单次实验即需要调动成百上千台服务器运行数个月才有结果。所以通过人工实验选择参数个数是不可行的,必须寻找合适的方法来解决模型选择的问题。

近年来,高斯过程(Gaussian process,GP)\cite{rasmussen2006gaussian}、狄利克雷过程(Dirichlet process,DP)\cite{TEH:06,teh2010dirichlet}等随机过程,由于其具备的特殊性质,可以用于建立能够根据数据分布自适应调整参数个数的模型,这类模型被称为非参数贝叶斯模型(Bayesian nonparametric models)\cite{gershman2012tutorial}。其中的狄利克雷过程具有良好的聚类性质,其聚类成分个数可以随着数据的变化而改变,从而被广泛的应用于建模含有混合成分的非参数模型,并在许多实际应用中都得到了可观的结果\cite{GHO:2012}。本文将对这一过程的相关模型和算法进行详细深入的研究,并应用在广播新闻故事分割和类音素这两个语言和语音处理任务中,取得了良好的效果。

\subsection{广播新闻故事分割}
广播新闻故事分割是指在新闻广播的音视频流上切分为出多个新闻故事的任务\cite{hauptmann1998story,JA:02},它是主题分割任务的一种特殊形式。主题分割泛指一系列需要对序列数据按照其主题语义进行切分的任务。对于广播新闻的检索系统,从整段新闻中分割出具有语义相关性质的小段落是非常重要的,目前,面对日益增长的海量数据,如果使用人工的方法来标注,工作量巨大,一般是不可能完成的。因此,寻找到合适的自动分割方法十分关键。除了用于新闻检索中,这一任务也可以进一步推广到一般的视频(泛指包含音频的视频)检索任务中。如今网络视频数据爆炸性增长,单个视频文件可能长达几十分钟或者数小时,此时必须对其进行有效地切分以提供更加细粒度的检索能力。另外,通过对视频进行有效地分割,可以帮助对视频进行进一步的分析和理解,这是其他许多任务不可缺少的环节。

对于这个任务,许多经典的方法都是基于词汇黏合(lexical cohesion)关系,即假设每个故事内的不同句子其用词是近似的,而不同故事之间的用词是有差异性的,这样将文档划分为黏合度较高的片段即可得到故事的边界。其中的典型代表是Hearst提出的TextTiling方法\cite{HEA:97,BANE:06,Wang:10},这一方法首先将句子转化为词频向量表示,然后利用一些相似度测度函数来考察相邻句子间的相似度,由于边界两侧的句子分属不同故事,其用词差距会导致相似度较低,因此,可以用相似度曲线上的谷点来检测边界。不过,通过寻找局部最小值来检测边界的结果只是局部最优的,Malioutov等人提出的最小割算法(Mincut)\cite{MAL:06},利用一个动态规划方法来寻找全局最优的故事边界。

另外,近几年基于主题模型的方法被应用在这一任务中,这类方法利用一系列主题模型如潜在语义分析(latent semantic analysis,LSA)、概率潜在语义分析(probability latent semantic analysis,PLSA)、隐狄利克雷分配(latent dirichlet allocation,LDA)等\cite{hofmann1999probabilistic,blei2003latent},先从含有边界信息的语料中学习出一组主题,然后将待分割语料的词频表示映射的这组主题表示上,从而得到一个更低维度的稀疏主题表示,进而利用textiling或者Mincut等方法在主题表示上进行进一步分割\cite{CHOI:01,HAL:08,CHI:12,LU:2011}。另外,Zheng和Lu等人从利用流形学习的方法,可以从语料中学习到更加适合分割的表示\cite{lu2013broadcast,xie2012laplacian}。这些方法相比于直接利用词频表示的方法,获得了更好的分割结果。但是这类方法需要标注好的语料作为获取主题模型的训练集,是一种有监督的方法。

然而基于Mincut方法都需要手工给定分段个数,本文中提出一种新的方法,通过增加一个依赖于距离相关的中国餐馆过程,为广播新闻故事分割任务建立了一个非参数贝叶斯模型,从而克服了这个缺点。

\subsection{类音素分割}
类音素分割任务是指将一段音频流分割为类音素单元流,这里的类音素是指一些类似于语音学中音素(phoneme),用于构成语音的最小单元,所以也称声学单元分割(acoustic unit segmentation)或者音素分割(phoneme segmentation)。对于许多传统的语音处理任务,如语音识别与合成,都是基于已经标注好的语料进行的,因此并不需要进行类音素分割。然而目前互联网数据库拥有海量的未标注语音资源,如果利用诸如深度神经网络的模型,可以通过这些数据自动学习到效果好的模型\cite{bengio2009learning},但是,对这些语音的标注需要大量的人力,人工进行标注是不现实的。因此,利用无监督的方法处理海量语音数据的研究非常必要。另外,许多语言,如一些非洲部落的语言,甚至未被语言学家研究过,如果可以找到一种无监督的框架,能够方便的推广到对各种语言的建模,就可以更好地推进对这些语言的研究。目前这一类问题被称为低资源语音研究(low resource)\cite{park2005towards,glass2012towards},意指利用最低限度的资源对语音进行无监督的研究,是目前语音领域研究的热点之一。而对低资源语音研究中,从特征层面往上,第一个任务便是子词级相关研究,即从原始的声音特征流中发现不同的音素,作为语音的最小组成单元,因此,音素分割具有重要的研究意义。

在音素分割的研究中,常用声学变化的峰值作为可能的边界,许多方法利用不同的声学变化准则进行分割\cite{aversano2001new,estevan2007finding,dusan2006relation}。Qiao引入了失真率的概念并以此为基础定义了代价函数,通过层次聚类算法不断地将相邻的语音帧合称为一个片段,得到最终的分割\cite{qiao2008unsupervised}。Scharenborg分析了这一自下而上的方法的局限性,利用自上而下的信息进行分割,来弥补仅仅使用声学变化为线索的不足\cite{scharenborg2010unsupervised}。本文以文献\cite{torbati2013speech,lee2012nonparametric}为参考,从概率的角度进行建模,将音素的分割和聚类统一起来,并针对音素个数未知的问题,建立非参数模型进行研究。

\section{本文主要工作和创新点}
本文旨在利用概率模型建模序列数据以解决分割问题。为了解决传统概率模型面临的模型选择问题,本文引入狄利克雷过程先验,针对故事分割和类音素分割两类序列分割问题建立了不同的非参数贝叶斯模型,并进行了实验验证与分析。本文的主要工作概括如下:
\vspace{-2pt}
\subparagraph{1.关于概率图模型理论的研究}
概率图模型作为描述现象的一种方法,广泛应用于各个领域。本文研究了基本的概率图建模方法,包括常见的概率分布,概率图模型的表示方法,以及相关的推断算法。深入探讨了如何用概率图模型描述数据的不同性质,以及如何从简单的模型扩展到复杂的模型。
\vspace{-2pt}
\subparagraph{2.关于非参数贝叶斯模型理论的研究}
传统的参数模型面临着模型选择问题,如果参数选择不当,会出现过拟合或者欠拟合的问题。非参数模型则根据数据来调整参数个数,从而解决模型选择的问题。本文对一类常用的非参数模型-狄利克雷过程进行了深入的研究。研究包括狄利克雷过程,狄利克雷过程混合(Dirichlet process mixture,DPM)以及分层狄利克雷过程(hierarchical Dirichlet process,HDP)模型。并深入讨论了这类模型在基于stick-breaking构造和中国餐馆过程(Chinese restaurant process,CRP)构造下的表示以及其在不同构造下对应的推断算法。对于中国餐馆过程,本文还研究了一种其对应的更一般的形式,称为依赖于距离的中国餐馆过程(distance dependent CRP,dd-CRP)。另外,本文还研究了将分层狄利克雷过程用于构造无限状态隐马尔科夫模型。
\vspace{-2pt}
\subparagraph{3.故事分割任务的非参数方法建模}
传统的基于Mincut的故事分割方法使用余弦相似度和交叉熵函数来度量词汇黏合度,然后再根据经验设计出函数来度量故事内聚性,这一方法对于不同问题需要设计不同的函数。本文从概率角度出发,利用贝叶斯模型对广播新闻故事分割任务进行建模,得到一种基于联合似然的内聚性度量。
另外,新闻的实际故事个数是未知,而基于MinCut框架的算法需要给定切分个数,这是基于MinCut框架的相关算法的一个缺点。本文提出一种新的方法,通过增加一个依赖于距离的中国餐馆过程,为广播新闻故事分割任务建立了一个非参数贝叶斯模型,从而解除了这一限制,并通过实验验证了这种方法的优势。
\vspace{-2pt}
\subparagraph{4.音素分割任务的非参数方法建模}
类音素分割和故事分割相比,除了在任务领域和数据特征上的不同,其最主要的区别在于,类音素分割任务中许多类音素单元是重复出现的,而故事分割却不具备这一性质。因此,本文为类音素分割任务建立了不同于故事分割的模型。对于类音素分割的建模,要考虑语音数据的时序性,即同一个语音单元对应的连续多帧是相似的,还需要考虑到音素之间的转换的统计性。本文用隐马尔科夫模型建模帧间的时序关系,用自跳转描述音素的持续性,用互跳转描述音素之间的转换。然而,隐马尔科模型需要给定模型中的状态个数,由于实际类音素单元的个数未知,本文将分层狄利克雷过程作为隐马尔科夫模型的先验,通过建立非参数模型,自适应的发现合适的状态个数。另外,本文还引入一个粘滞参数来建模音素的持续性,使得模型具有一定的自跳转偏执。这种方法称为基于sticky-HDPHMM模型的方法。另外,类音素分割任务往往是类音素发现任务的子任务,即对分割得到的结果还需要再进行进一步的聚类。本文的方法将这两个过程统一起来,其分割和聚类的过程被蕴含在模型的迭代推断过程之中。实验结果验证了本文方法的有效性。

\vspace{25pt}
本文主要的创新点如下:

1.本文通过建立概率生成模型来描述新闻广播故事的生成过程,得到的内聚性度量拥有很好的物理解释,并且具有更好的切分效果。进一步,针对广播新闻故事分割的故事个数未知的问题,本文提出了一种基于依赖于距离的中国餐馆过程的方法,有效的解决了这一问题。

2.通过将分层狄利克雷过程作为隐马尔科夫模型的先验,为音素分割任务建立了一个含有无限状态的隐马尔科夫模型,称之为sticky-HDPHMM模型,解决了音素个数未知的问题,并取得了良好的音素分割效果。


\section{本文的组织结构}
\subparagraph{第一章} 综述了论文的研究背景和内容,并给出了本文的主要工作,创新点以及组织结构。
\subparagraph{第二章} 回顾了概率模型的基础知识,包括基本的概率分布,有向图和无向图模型的表示,置信传播算法,基于Gibbs采样的推断。另外,还研究了几种常见的概率模型以及其之间的关系。
\subparagraph{第三章} 研究了狄利克雷过程的相关模型,对基于不同构造的推断和参数学习方法进行了研究。
\subparagraph{第四章} 首先回顾了故事分割任务的非概率模型方法,然后提出了利用概率模型建模的新模型,并给出了实验的结果和分析。
\subparagraph{第五章} 首先介绍并分析了类音素分割任务,介绍了无限状态隐马尔科夫模型,以及为其增加自跳转偏置变量的改进和相关推断算法,并给出了实验结果和分析。
\subparagraph{第六章} 总结了全文并给出了一些下一步需要研究和解决的问题。